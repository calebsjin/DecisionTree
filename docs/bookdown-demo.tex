\documentclass[]{book}
\usepackage{lmodern}
\usepackage{amssymb,amsmath}
\usepackage{ifxetex,ifluatex}
\usepackage{fixltx2e} % provides \textsubscript
\ifnum 0\ifxetex 1\fi\ifluatex 1\fi=0 % if pdftex
  \usepackage[T1]{fontenc}
  \usepackage[utf8]{inputenc}
\else % if luatex or xelatex
  \ifxetex
    \usepackage{mathspec}
  \else
    \usepackage{fontspec}
  \fi
  \defaultfontfeatures{Ligatures=TeX,Scale=MatchLowercase}
\fi
% use upquote if available, for straight quotes in verbatim environments
\IfFileExists{upquote.sty}{\usepackage{upquote}}{}
% use microtype if available
\IfFileExists{microtype.sty}{%
\usepackage{microtype}
\UseMicrotypeSet[protrusion]{basicmath} % disable protrusion for tt fonts
}{}
\usepackage{hyperref}
\hypersetup{unicode=true,
            pdftitle={Decision Tree},
            pdfauthor={Caleb Jin},
            pdfborder={0 0 0},
            breaklinks=true}
\urlstyle{same}  % don't use monospace font for urls
\usepackage{natbib}
\bibliographystyle{apalike}
\usepackage{longtable,booktabs}
\usepackage{graphicx,grffile}
\makeatletter
\def\maxwidth{\ifdim\Gin@nat@width>\linewidth\linewidth\else\Gin@nat@width\fi}
\def\maxheight{\ifdim\Gin@nat@height>\textheight\textheight\else\Gin@nat@height\fi}
\makeatother
% Scale images if necessary, so that they will not overflow the page
% margins by default, and it is still possible to overwrite the defaults
% using explicit options in \includegraphics[width, height, ...]{}
\setkeys{Gin}{width=\maxwidth,height=\maxheight,keepaspectratio}
\IfFileExists{parskip.sty}{%
\usepackage{parskip}
}{% else
\setlength{\parindent}{0pt}
\setlength{\parskip}{6pt plus 2pt minus 1pt}
}
\setlength{\emergencystretch}{3em}  % prevent overfull lines
\providecommand{\tightlist}{%
  \setlength{\itemsep}{0pt}\setlength{\parskip}{0pt}}
\setcounter{secnumdepth}{5}
% Redefines (sub)paragraphs to behave more like sections
\ifx\paragraph\undefined\else
\let\oldparagraph\paragraph
\renewcommand{\paragraph}[1]{\oldparagraph{#1}\mbox{}}
\fi
\ifx\subparagraph\undefined\else
\let\oldsubparagraph\subparagraph
\renewcommand{\subparagraph}[1]{\oldsubparagraph{#1}\mbox{}}
\fi

%%% Use protect on footnotes to avoid problems with footnotes in titles
\let\rmarkdownfootnote\footnote%
\def\footnote{\protect\rmarkdownfootnote}

%%% Change title format to be more compact
\usepackage{titling}

% Create subtitle command for use in maketitle
\providecommand{\subtitle}[1]{
  \posttitle{
    \begin{center}\large#1\end{center}
    }
}

\setlength{\droptitle}{-2em}

  \title{Decision Tree}
    \pretitle{\vspace{\droptitle}\centering\huge}
  \posttitle{\par}
    \author{Caleb Jin}
    \preauthor{\centering\large\emph}
  \postauthor{\par}
      \predate{\centering\large\emph}
  \postdate{\par}
    \date{9-27-2019}

\usepackage{booktabs}
\usepackage{amsthm}
\makeatletter
\def\thm@space@setup{%
  \thm@preskip=8pt plus 2pt minus 4pt
  \thm@postskip=\thm@preskip
}
\makeatother

\begin{document}
\maketitle

{
\setcounter{tocdepth}{1}
\tableofcontents
}
\hypertarget{Foreword}{%
\chapter*{Foreword}\label{Foreword}}
\addcontentsline{toc}{chapter}{Foreword}

I am \href{https://www.sjin.name/}{Caleb Jin}. This is my note of \textbf{decision tree}. When I try to grasp a statistical method, I'd like to wirte down every details about it that I am able to. I mainly use \textbf{An Introduction to Statistical Learning with Applications in R} \citep{James2014} and \textbf{The Elements of Statistical Learning} \citep{hastie2008}. Due to my limited statistics knowledge, if making any mistakes, I sincerely expect you guys can email to me. My email address is \href{mailto:jinsq@ksu.edu}{\nolinkurl{jinsq@ksu.edu}}. Appreicate!

\newcommand\uy{{\bf y}}
\newcommand\uX{{\bf X}}
\newcommand\ux{{\bf x}}

\hypertarget{basic}{%
\chapter{The Basic of Decision Tree}\label{basic}}

Let's start with a simple model setting. Consider we have a continuous response variable \({\bf y}=(y_1,y_2, \ldots, y_n)\) and 2 predictors \({\bf X}= ({\bf x}_1,{\bf x}_2)\in \mathbb{R}^{n\times 2}\).

The decision tree starts with splitting a predictor, say \({\bf x}_1\),

\begin{itemize}
\item
  \begin{enumerate}
  \def\labelenumi{\arabic{enumi})}
  \tightlist
  \item
    We partition \({\bf x}_1\) into two distinct regions \(R_1(1,s) = \{{\bf x}_1|{\bf x}_1<s\}\) and \(R_2(1,s) = \{{\bf x}_1|{\bf x}_1\geq s\}\).
  \end{enumerate}
\item
  \begin{enumerate}
  \def\labelenumi{\arabic{enumi})}
  \setcounter{enumi}{1}
  \tightlist
  \item
    As all observations are divided into two regions \(R_1(1,s)\) or \(R_2(1,s)\), then we make the same prediction with \[
    \hat y_{R_1}=\frac{1}{n_1}\sum_{i:x_{i1}\in R_1} y_i,\]
    \[\hat y_{R_2}=\frac{1}{n_2}\sum_{i: x_{i1}\in R_2} y_i.\]
  \end{enumerate}
\end{itemize}

\textbf{The question is how to determine \(s\)}.

From the step 1), we get \(R_1(1,s)\) and \(R_2(1,s)\) by spliting \({\bf x}_1\), for example. We hope this splitting can maximize sum of squares between regions and minimize sum of squares within regions of \({\bf y}\). As total sum of squares of \({\bf y}\) is fixed, maximium of sum of squares between regions is equivalent to minimum of sum of squares within regions. This lead us to consider a classic criterion, residual sum of squares (RSS):
\[
RSS = \sum_{j=1}^{2}\sum_{i:x_{i1}\in R_j(1,s)}(y_i - \hat y_{R_j})^2.
\]

\bibliography{book.bib,packages.bib}


\end{document}
